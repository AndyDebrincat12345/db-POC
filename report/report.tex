\documentclass[11pt,a4paper]{article}
\usepackage[utf8]{inputenc}
\usepackage[margin=1in]{geometry}
\usepackage{booktabs}
\usepackage{longtable}
\usepackage{array}
\usepackage{xcolor}
\usepackage{graphicx}
\usepackage{hyperref}
\usepackage{enumitem}
\usepackage{tabularx}
\usepackage{amssymb}
\usepackage{eso-pic}
\usepackage{tikz}

% Define simple color for the side line
\definecolor{eyyellow}{RGB}{255, 237, 55}

% Add side line to every page
\AddToShipoutPicture{%
  \begin{tikzpicture}[remember picture,overlay]
    % Yellow line on the far left
    \draw[line width=8pt, color=eyyellow] 
      (current page.south west) -- (current page.north west);
  \end{tikzpicture}
}

\title{Database Migration Tools Comparison}
\author{Graham Pellegrini}
\date{\today}

\begin{document}

\maketitle

\tableofcontents
\newpage

\section{Executive Summary}

This document provides a comparison of three leading database migration and schema management tools: Bytebase, Liquibase, and Redgate.
Each tool offers distinct approaches to database versioning, migration management, and team collaboration, catering to different organizational needs and technical requirements.

\section{Introduction}
The testing was conducted using Microsoft SQL Server and MySQL databases, focusing on the  capabilities of the freely available versions of each tool.\\
\textbf{What we tested:}
\begin{itemize}
    \item All tools used the same SQL migration scripts
    \item Tested connections to both SQL Server and MySQL
    \item Evaluated free/community editions only
\end{itemize}

\textbf{Tools compared:}
\begin{itemize}
    \item \href{https://bytebase.com}{\textbf{Bytebase Community}}
    \item \href{https://liquibase.org}{\textbf{Liquibase Open Source}}
    \item \href{https://www.red-gate.com/products/sql-development/sql-compare/}{\textbf{Redgate Free Edition}}
\end{itemize}

\newpage

\section{Bytebase}

Modern web-based tool with a clean interface.
Works like a database version of GitHub, great for teams who want visual collaboration.


\begin{itemize}
    \item Easy-to-use web interface
    \item Connects to multiple database types
    \item Tracks who changed what and when
    \item Admin and User permissions configuration
    \item Free for up to 20 users and 5 databases
\end{itemize}

\textbf{Setup:}

To set up Bytebase, you need to have Docker installed and running on your machine.
Once Docker is ready, you can pull the Bytebase Docker image and start the container.
The web interface will be accessible through your browser at the specified port.
Bytebase code works using a simple folder structure with SQL migration files and a single YAML configuration file \texttt{bytebase-config.yaml} to manage database changes.
The tool then provides a web service accessible through your browser, from which you can manage your database migrations.\\

\textbf{Testing Outcomes:}

The added dependency on Docker means that if Docker isn't running, Bytebase can't function.
This was the main hurdle faced during testing and was an additional learning that had to be tackled.
Bytebase then showed its potential by providing a robust migration tracking system and an intuitive web interface for managing database changes.
The login process when debugging was tedious as it would constantly prompt for a new admin login and reusing email addresses was not possible.
Bytebase effectively managed redundant migrations and provided clear visibility into the migration history.
That is processing time for migrations where no changes were made was minimal, showcasing Bytebase's efficiency.\\

\textbf{Overview:}

Bytebase code was simple to learn and interface seems great for teams, the free version is limited for larger projects.
Migration tracking is excellent, and the web interface is intuitive.
Version control is not as robust as traditional Git-based systems, which may be a drawback for some teams.hat values simplicity, are comfortable with Docker containers, and don't mind the additional infrastructure setup requirements.\\

\newpage
\section{Liquibase}

Command-line tool that's been around for years. Uses text files to track database changes, known to be used in automation.

\begin{itemize}
    \item Supports 40+ database types
    \item Works with any CI/CD pipeline
    \item Completely free with no user limits
    \item Flexibility in changelog code formats
\end{itemize}

\textbf{Setup:}

Liquibase supports three changelog formats: XML, YAML, and SQL files.
XML files were used for our migrations because they provide better structure validation and error checking compared to plain SQL.
The tool uses a master changelog file that references all individual migration files, with configuration managed through a liquibase.properties file. 
Unlike the other tools, Liquibase can be fully accessed and used directly from code/command line without requiring separate applications.
This implementation required manually downloading the correct database libraries (JDBC drivers) and adjusting existing ones to work with our Python virtual environment.
This was a very tedious process, especially getting Liquibase to properly identify file paths.
The tool is heavily dependent on where you run commands from.
Path configurations in the properties file and relative references in XML files must be precisely aligned, making it fragile when running from different directories.\\

\textbf{Testing Results:}

Worked with MySQL successfully and showed consistent performance in automated testing (3-4 second execution times).
Could not connect to Microsoft SQL Server due to a TCP/IP network requirement restricted by corporate policy. 
Causing the automated system to skip Liquibase for SQL Server migrations entirely.
When it worked, Liquibase demonstrated reliable execution with proper status reporting ("Database schema is up to date", "No new changesets to execute").
All operations generate detailed log files, making it excellent for debugging and audit trails.
This file-based logging is particularly useful for automated environments and troubleshooting.\\

\textbf{Overview:}

Liquibase was firstly free and open-source, but has since introduced paid plans with additional features.
Out of the box it is complex to learn but powerful for automation on various database platforms.
Integration is fully with code and CI/CD pipelines so it can be easily incorporated into existing workflows.
There is no web interface, which may be a drawback for teams that prefer visual tools.
It should be chosen if you need maximum database support and command-line integration is needed.\\

\newpage
\section{Redgate}

Application based tool that provides visual tools focused on SQL Server.

\begin{itemize}
    \item Excellent visual comparison tools
    \item Built specifically for SQL Server
    \item Easy drag-and-drop interface
    \item Professional-grade features
\end{itemize}

\textbf{Setup:}

Unlike Bytebase's web interface or Liquibase's command-line approach, Redgate requires downloading and installing a separate Windows desktop application.
The "visual" part means it shows database schemas side-by-side in a graphical interface where you can see differences highlighted in colors, click to navigate between objects, and use drag-and-drop to synchronize changes.
It's a standalone desktop application, not integrated with your development environment.
Must download and install the desktop application from Redgate's website.
Cannot be installed via pip or used directly within VS Code like Python packages.
The code initially attempted was a simulation of the visual interface, simply using SQL queries to compare schemas and generate migration scripts.\\

\textbf{Testing Results:}

Connected successfully to SQL Server and showed consistently fast execution times in automated testing (0.6 seconds for SQL Server, 0.5-0.7 seconds for MySQL).
Demonstrated reliable connectivity and quick processing through PowerShell integration across both database platforms.\\

\textbf{Overview:}
This tool offers best-in-class support for SQL Server with an intuitive desktop interface, visual comparisons, and professional support, but it’s costly for full features, limited to SQL Server, and only available as a Windows desktop app.
Choose it if you work primarily with SQL Server, prefer desktop tools, and have the budget for premium software.

\newpage
\section{Detailed Comparison}

\subsection{Licensing and Cost (Free Versions Focus)}

\begin{table}[h!]
\centering
\begin{tabularx}{\textwidth}{|X|X|X|X|}
\hline
\textbf{Aspect} & \textbf{Bytebase Community} & \textbf{Liquibase Open Source} & \textbf{Redgate Free Tools} \\
\hline
License Type & Free Community Edition & Open Source (Apache 2.0) & Limited Free Edition \\
\hline
User Limitations & Up to 20 users & Unlimited & Single user \\
\hline
Database Instances & Up to 5 instances & Unlimited & Limited \\
\hline
Enterprise Features & Limited RBAC, Basic audit & None & Very limited \\
\hline
Support & Community only & Community only & Limited \\
\hline
Upgrade Path & Seamless to Enterprise & Liquibase Pro available & Full Redgate suite \\
\hline
\end{tabularx}
\caption{Free Version Licensing Comparison}
\end{table}

\subsection{Database Support}

\begin{table}[h!]
\centering
\begin{tabularx}{\textwidth}{|X|X|X|X|}
\hline
\textbf{Database} & \textbf{Bytebase} & \textbf{Liquibase} & \textbf{Redgate} \\
\hline
PostgreSQL & \checkmark & \checkmark & Limited \\
\hline
MySQL & \checkmark & \checkmark & Limited \\
\hline
SQL Server & \checkmark & \checkmark & \checkmark (Primary) \\
\hline
Oracle & \checkmark & \checkmark & Limited \\
\hline
MongoDB & \checkmark & \checkmark & No \\
\hline
SQLite & \checkmark & \checkmark & No \\
\hline
Total Supported & 15+ & 40+ & 10+ \\
\hline
\end{tabularx}
\caption{Database Platform Support}
\end{table}

\subsection{Feature Comparison}

\begin{longtable}{|p{3cm}|p{3cm}|p{3cm}|p{3cm}|}
\hline
\textbf{Feature} & \textbf{Bytebase} & \textbf{Liquibase} & \textbf{Redgate} \\
\hline
\endfirsthead

\hline
\textbf{Feature} & \textbf{Bytebase} & \textbf{Liquibase} & \textbf{Redgate} \\
\hline
\endhead

Web Interface & \checkmark & Limited & \checkmark \\
\hline
Command Line & \checkmark & \checkmark & \checkmark \\
\hline
Version Control & Git Integration & File-based & Git/TFS Integration \\
\hline
Rollback & \checkmark & \checkmark & \checkmark \\
\hline
Schema Drift & \checkmark & Limited & \checkmark \\
\hline
Data Migration & \checkmark & \checkmark & \checkmark \\
\hline
Multi-environment & \checkmark & \checkmark & \checkmark \\
\hline
RBAC & \checkmark & Enterprise only & \checkmark \\
\hline
API Access & \checkmark & \checkmark & Limited \\
\hline
Audit Trail & \checkmark & Basic & \checkmark \\
\hline
\caption{Feature Comparison Matrix}
\end{longtable}

\newpage
\section{Conclusion}

\textbf{Quick Summary:}

\begin{itemize}
    \item \textbf{Bytebase} - Best for modern, easy to use interface with strong team collaboration features.
    \item \textbf{Liquibase} - Best for developers who need maximum database support and automation.
    \item \textbf{Redgate} - Best for SQL Server environments with budget for premium tools. With the want to use visual application features (Microsoft Software Style).
\end{itemize}

The testing conducted was not exhaustive.
A consistent set of migrations and database commands was used across all tools for comparison.
However, varying database configurations and workloads were not considered.
These factors may significantly affect performance and usability in real-world scenarios.

Some tool features have not been fully explored or evaluated.
For example, user-level interactions with Bytebase were not tested, as co-workers were unavailable to assist.

This report serves as a baseline overview of available features and highlights areas that may warrant further investigation.
The full scope of testing was limited due to time constraints and competing deadlines.

\end{document}